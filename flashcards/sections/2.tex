\section{Überblick}

\begin{itemize}[leftmargin=*]
\item Schach eines am längsten erforschten Themen der KI
\item Pioniere der KI wie Alan Turing oder Claude Shannon waren selber Schachspieler
\item Betrachteten Schachspiel als ideales Modell für maschinelles Lernen
\item Forschung hauptsächlich auf Gebiet der Schachengines
\item Schachengines sind Schachprogramme
\item Ziel ist Optimierung der Spielstärke
\item Heute liegt Spielstärke der Engiens weit über der der besten Profispieler
\item Hohes Spielniveau führt zu Intransparenz von Zügen bzw. Zugfolgen
\item Professionelle Spieler oder Kommentatoren werden benötigt, um Ideen hinter Zügen des Computers zu verstehen
\item Auch deren Analysen sind oft nicht korrekt
\end{itemize}

\newpage

\begin{itemize}[leftmargin=*]
\item Forschungsfrage versucht, Problem der Intransparenz von Zügen bzw. Zugfolgen zu lösen
\item Menschliche Kommentatoren durch virtuelle ersetzt werden
\item Selbstverständlich nicht nur Partien von Schachengines, sondern auch menschliche Partien kommentieren
\item Kann auch genutzt werden um eigene Spielstärke zu verbessern
\end{itemize}

\newpage


\begin{itemize}[leftmargin=*]
\item Um Forschungsfrage zu lösen, wird in zwei Teile aufgeteilt
\item Zum einen Informationsbeschaffung
\item Zum anderen Kommentarerstellung
\item Informationsbeschaffung befasst sich \textit{woher} Informationen kommen die zum übersetzen nötig sind
\item Kommentargenerierung nimmt Informationen und beschäftigt \textit{wie} und \textit{was} für Kommentare erzeugt werden
\item Für ersten Teil bereits erwähnte Schach Engine verwendet
\item Für zweiten Teil Encoder-Decoder-Modell verwendet
\end{itemize}