\section{Überblick}

Schach gehört zu einem der am längsten erforschten Themen der Künstlichen Intelligenz. Das liegt vor allem daran, dass Pioniere der KI wie Alan Turing Schachspieler waren und das Schachspiel als ideales Modell für maschinelles Lernen betrachteten.\\

Damals wie heute wurde hauptsächlich auf dem Gebiet der Schachengines geforscht. Das sind Schachprogramme. Das Ziel der Forschung ist es, die Spielstärke zu optimieren. Heute liegt die Spielstärke der Engiens weit über der der besten Profispieler.\\

Ein Problem ist, dass dieses hohe Spielniveau zu einer Intransparenz von Zügen bzw. Zugfolgen führt. Das bedeutet, dass man professionelle Spieler oder Kommentatoren braucht, um die Ideen hinter den Zügen des Computers zu verstehen, und selbst deren Analysen sind nicht immer korrekt.\\

Meine Forschungsfrage "Wie kann maschinelles Lernen genutzt werden, um Kommentare zu Schachpartien zu generieren?" versucht, dieses Problem der Intransparenz von Zügen zu lösen, indem versucht wird, menschliche Kommentatoren durch virtuelle zu ersetzen.

\newpage

Um die Forschungsfrage zu lösen, teilen wir sie in zwei Teile. Zum einen die Informationsbeschaffung und die Kommentarerstellung. Die Informationsbeschaffung befasst damit \textit{woher} wir Informationen bekommen die zum übersetzen nötig ist. Die Kommentargenerierung nimmt diese Informationen und beschäftigt dann damit \textit{wie} und \textit{was} für Kommentare erzeugt werden sollen.\\

Für den ersten Teil wird eine bereits erwähnte Schach Engine verwendet, für den zweiten Teil das sogenannte Encoder-Decoder-Modell.\\

