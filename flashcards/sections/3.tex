\section{Schach Engine}

\begin{itemize}[leftmargin=*]
\item Engines sind Schachprogramme
\item Müssen Reihe von Anforderungen erfüllen
\begin{itemize}
\item Brett Darstellung
\item Zugsuche
\item Positionsbewertung
\end{itemize}
\item Anforderungen stellen Informationen bereit die später benötigt werden
\end{itemize}

\newpage

\subsection{Brett Darstellung}

\begin{itemize}[leftmargin=*]
\item Für Computer muss Schachbrett und Figuren in verarbeitbare Form gebracht werden
\item Standardimplementierung Bitboards
\item Jedes Feld kann Wert 0 oder 1 haben
	\begin{itemize}[leftmargin=*]
		\item 0: Figur belegt Feld nicht
		\item 1: Figur belegt Feld
	\end{itemize}
\item Für jeden Figurentyp, Figurenfarbe und für die Rochade werden Bitboards erstellt.
\item Rochade ist spezieller Spielzug, um König in sichere Position zu bringen
\item Anzahl Bitboards kann je nach Implementierung variieren
\item Hier insgesamt 16
\end{itemize}

\newpage

\begin{itemize}[leftmargin=*]
\item Mögliche Implementierung sind $8 \times 8$ Arrays
\item Alterantiv eindimensionalen Array oder als Objekte
\item Beispiel: Bitboard für weiße Bauern
\item Durch "Übereinanderlegen" der Bitboards können Züge mit Hilfe logischer Operationen berechnet werden
\item Vorteil ist, dass Darstellung als Input für neuronale Netze verwendet werden kann
\end{itemize}

\newpage

\subsection{Zugsuche und Positionsbewertung}

\begin{itemize}[leftmargin=*]
\item Schach ist bisher ungelöstes Spiel
\item Nicht bekannt, ob es Methode gibt, die immer zum Sieg oder, wenn sie von beiden Seiten angewendet wird, zum Unentschieden führt
\item Qualität Zugsuche und Positionsbewertung immer Basis der Programmierung und Rechenleistung des Systems
\item Bekannte Implementierungen, ohne maschinelles Lernen:
\begin{itemize}[leftmargin=*]
\item MiniMax-Algorithmus bzw. Erweiterung Alpha-Beta Pruning für Zugsuche
\item handgeschriebene Bewertungsfunktion
\end{itemize}
\item Zeigen Methode basierend auf maschinellem Lernen
\begin{itemize}[leftmargin=*]
\item Monte-Carlo-Suchbaum für Zugsuche
\item Neuronales Netz für Positionsbewertung
\end{itemize}
\item Ansatz basiert auf AlphaZero-Engine
\item War in der Lage, stärksten bekannten Engines zu schlagen, die ohne maschinelles Lernen arbeiten
\end{itemize}

\newpage

Wie genau der Ansatz funktioniert erkläre ich an einer Abbildung.

\begin{enumerate}[leftmargin=*]
\item Haben Positon gegeben, die in BitBoard Darstellung vorliegt
\item Codeierte Position wird als Input für neuronales Netzwerk verwendet, welche die Position bewertet. Als Output gibt es Bewertung (hier $v$) und Wahrscheinlichkeiten für folgende Züge (hier $p_1$ bis $p_3$ für drei Züge) aus.
\item Monte Carlo Suchbaum, generiert alle möglichen legalen Züge
\item Der ursprüngliche Position wird dann die Bewertung $v$ zugeordnet und die berechneten Zügen bekommen die Zugwahrscheinlichkeiten $p$. Der Suchbaum wählt dann Pfade, teils zufällig, aus, simuliert Partien und bewerte all diese Positionen und Züge mit Hilfe des Neuronalen Netzwerks.
\item Für alle Pfade werden Gesamtstatistiken erstellt und der Pfad mit der besten Gesamtstatistik wird gespielt.
\end{enumerate}

\newpage

Damit das Neuronale Netzwerk zuverlässige Bewertungen erzeugt muss es trainiert werden. In traditionellen Engines wurde Trainingsdaten, aus Datenbanken, welche Profipartien speichern, benutzt. Die Daten können allerdings auch durch Selbstspiel generiert werden.\\

Dabei exitsiert Anfangs eine Engine besteht aus den Spielregeln, einem untrainiertes neuronales Netz und der MCTS-Algorithmus\\

Um das Netzwerk zu trainieren werden nun 4 Schritte immer wieder wiederholt.

\begin{enumerate}[leftmargin=*]
\item Das Programm spielt gegen sich selbst, zeichnet jede Partie auf und weißt dann einer Partie verloren, unentschieden oder gewonnen zu.
\item Das Neuronale Netzwerk wird geklont und bekommt die Partien, zusammen mit dem Wert, um die Parameter anzupassen
\item Das Programm mit dem neuem Neuronale Netzwerk spielt dann gegen das alte Programm
\item Das Programm das gewinnt wird dann ausgewählt und es geht von vorne los
\end{enumerate}\

Durch diese Trainingsmethode konnte die damals beste Engine \textit{Stockfish}, in nur 4 Stunden übertroffen werden.

\newpage

Aber welche Informationen haben wir nun von der Engine bekommen.

\begin{enumerate}[leftmargin=*]
\item Zum einen verarbeitbare Brettdarstellung
\item Bewertung ob Stellung gut oder schlecht ist
\item Zugwahrscheinlichkeiten
\item Zugpfade durch MCTS um Absicht hinter Strategie zu bestimmen oder Züge zu vergleichen
\end{enumerate}