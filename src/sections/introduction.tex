\section{Introduction}

In the mid-20th century, computer chess experienced its first breakthroughs thanks to the work of scientists like Alan Turing and Claude Shannon. Alan Turing, the pioneer of artificial intelligence, was convinced that games were an ideal model system for machine learning. \footnote{Vgl. Levy et al. 1982, p.44-45} This prediction has come true, and machine learning has grown to be an essential part of any chess engine today. Although chess engines have become a powerful tool, they have a lack of transparency regarding the moves they perform. Therefore, professional chess players and commentators are often needed to explain the intention of these moves. This dependence on human chess commentators can be a drawback, since moves found by computers can be misinterpreted and, above all, appear incomprehensible to non-professional chess players. In the following, I will address the question of how to overcome this intransparency using machine learning and create a virtual chess commentator that translates the engines intentions into human-understandable language.