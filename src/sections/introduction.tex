\section{Introduction}

In the mid-20th century, computer chess experienced its first breakthroughs thanks to the work of scientists like Alan Turing and Claude Shannon. Alan Turing, the pioneer of artificial intelligence, was convinced that games were an ideal model system for machine learning.\footnote{Cf. Levy et al. 1982, p. 44-45} This prediction has come true, and machine learning have proven to be an essential part of many chess engine today. In particular, recent projects such as AlphaZero, developed by DeepMind, show how efficient programs which use neural networks are in analyzing board games compared to traditionally used algorithms such as alpha-beta search.\footnote{See Silver et al. 2018 p. 1} Although chess engines have become a powerful tool, they have a lack of transparency regarding the moves they perform. Therefore, professional chess players and commentators are often needed to explain the intention of these moves. This dependence on human chess commentators can be a drawback, since moves found by computers can be misinterpreted and, above all, appear incomprehensible to non-professional chess players. In the following, I will explore the question of how this opacity can be overcome using a neural network to create a virtual chess commentator that uses a built-in chess engine to translate the engine's intentions into a language that humans can understand.