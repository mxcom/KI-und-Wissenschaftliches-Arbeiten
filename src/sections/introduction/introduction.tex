\section{Introduction}

In the mid-20th century, computer chess experienced its first breakthroughs thanks to the work of scientists like Alan Turing, Claude Shannon and John von Neumann.\footnote{Cf. \cite{keen-2009-history}, p. 3} Alan Turing, the pioneer of artificial intelligence, was convinced that games are an ideal model system for machine learning.\footnote{Cf. \cite{levy-newborn-1982}, pp. 44-45} This prediction has come true, and machine learning have proven to be an essential part of many chess engines today. In particular, recent projects such as AlphaZero, developed by DeepMind, have shown how efficient programs which use neural networks are in analyzing board games compared to traditionally used algorithms such as alpha-beta search.\footnote{Cf. \cite{alphazero-2018}, p. 1} Although chess engines have become a powerful tool, they have a lack of transparency regarding the moves they perform. Therefore, professional chess players and commentators are often needed to explain the intention of these moves. This dependence on human chess commentators can be a disadvantage, since moves found by computers can sometimes be misinterpreted or not understood at all. Especially for non-professional chess players, most moves played by professional players as well as computers are incomprehensible, since they do not have the appropriate experience. To solve this problem, in the following the question "how machine learning can be used to generate commentary for chess games" will be discussed. For this the construction of a chess engine is presented, which corresponds in its play and analysis strength to the today's engines. After a brief introduction on how to analyze the moves in depth, the construction of a virtual chess commentator is described, which uses the information gained by the analysis of the chess engine to create detailed comments on a given position and move.