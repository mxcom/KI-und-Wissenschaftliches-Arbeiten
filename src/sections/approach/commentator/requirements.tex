\subsubsection{Requirements}

The virtual chess commentator has the task of using certain information provided by the engine and translating it into a format that can be understood by humans. The requirements that the commentator should fulfill are the \textit{description of the current move}, the \textit{description of the move quality}, the \textit{comparison of moves}, the \textit{description of the move planning} and \textit{contextual game information}.\footnote{Cf. Jhamtani et al. 2018 p. 3}$^{,}$\footnote{Zang et al. 2019, pp. 3-4} These requirements come with two challenges that can have a strong impact on the quality of the generated description: (1) the number of moves and the resulting position considered, (2) the features on the basis of which the comments are generated. 


% In order to generate comments on these requirements, it is necessary to define certain features on the basis of which this is done. \cite{jhamtani-etal-2018-learning} uses features such as, \gls{threats} and position value, but as \cite{zang-etal-2019-automated} show, this information does not consider the following positions well enough, which affects the quality of comments. To solve this problem the requirements were divided into two categories: (1) requirements which need exactly one move to be describe, (2) requirements which need a number of moves to be describe. The second category is only an extension of the first, so the two categories differ in detail only in how many positions and moves they consider; the features included remain the same.

% A method which has shown significant success in the field of natural language processing are Recurrent Neural Networks (RNN).

% The process of generating the text commentary can be summarized into two upper parts, namely encoding and decoding. Encoding describes the extraction and formatting of certain features to make them easier to process, while decoding describes the translation of the encoded features into a text format.