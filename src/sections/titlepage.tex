\begin{titlepage}
% University Informations
\begin{flushleft}
Frankfurt University of Applied Sciences\\
Fachbereich 2: Informatik und Ingenieurwissenschaften\\
Informatik (B.Sc.)\\
\end{flushleft}

\vspace{3.5cm}

% Thesis Title
\begin{center}
\Large
\textbf{Generating Chess Commentary using\\Machine Learning}\\
%\large
% Subtitle
\end{center}

% Abstract
\begin{abstract}
This scientific seminar paper deals with the question how machine learning can be used to generate comments on chess games. On the one hand it is shown how a chess engine can be built to analyze games and on the other hand how the information provided by the engine can be used by a virtual chess commentator to generate comments on them. The engine is built by a neural network and a Monte-Carlo Tree Search algorithm and can be trained by self-play. The virtual chess commentator uses an encoder-decoder model which uses bidirectional LSTMs architecture.
\end{abstract}

\vspace{6cm}
	
% Lecturer Informations
\begin{flushright}
Lecturer: Konstantin Ernst\\
Course: Künstliche Intelligenz und wissenschaftliches Arbeiten\\
Winter Semester 22/23\\
\end{flushright}

\vspace{2cm}

% Your Informations
\begin{flushleft}
Submitted by:\\
Max Semdner\\
Matrikelnr.: 1294899\\
\href{mailto: max.semdner@stud.fra-uas.de}{max.semdner@stud.fra-uas.de}\\
\end{flushleft}

\end{titlepage}